\newif\ifshowsolutions
\showsolutionstrue
\documentclass{article}
\usepackage{listings}
\usepackage{amsmath}
%\usepackage{subfigure}
\usepackage{subfig}
\usepackage{amsthm}
\usepackage{amsmath}
\usepackage{amssymb}
\usepackage{graphicx}
\usepackage{mdwlist}
\usepackage[colorlinks=true]{hyperref}
\usepackage{geometry}
\usepackage{titlesec}
\geometry{margin=1in}
\geometry{headheight=2in}
\geometry{top=2in}
\usepackage{palatino}
\usepackage{mathrsfs}
\usepackage{fancyhdr}
\usepackage{paralist}
\usepackage{todonotes}
\setlength{\marginparwidth}{2.15cm}
\usepackage{tikz}
\usetikzlibrary{positioning,shapes,backgrounds}
\usepackage{float} % Place figures where you ACTUALLY want it
\usepackage{comment} % a hack to toggle sections
\usepackage{ifthen}
\usepackage{mdframed}
\usepackage{verbatim}
\usepackage[strings]{underscore}
\usepackage{listings}
\usepackage{bbm}
\rhead{}
\lhead{}

\renewcommand{\baselinestretch}{1.15}

% Shortcuts for commonly used operators
\newcommand{\E}{\mathbb{E}}
\newcommand{\Var}{\operatorname{Var}}
\newcommand{\Cov}{\operatorname{Cov}}
\newcommand{\Bias}{\operatorname{Bias}}
\DeclareMathOperator{\argmin}{arg\,min}
\DeclareMathOperator{\argmax}{arg\,max}

% do not number subsection and below
\setcounter{secnumdepth}{1}

% custom format subsection
\titleformat*{\subsection}{\large\bfseries}

% set up the \question shortcut
\newcounter{question}[section]
\newenvironment{question}[1][]
  {\refstepcounter{question}\par\addvspace{1em}\textbf{Question~\Alph{question}\!
    \ifthenelse{\equal{#1}{}}{}{ [#1 points]}: }}
    {\par\vspace{\baselineskip}}

\newcounter{subquestion}[question]
\newenvironment{subquestion}[1][]
  {\refstepcounter{subquestion}\par\medskip\textbf{\roman{subquestion}.\!
    \ifthenelse{\equal{#1}{}}{}{ [#1 points]:}} }
  {\par\addvspace{\baselineskip}}

\titlespacing\section{0pt}{12pt plus 2pt minus 2pt}{0pt plus 2pt minus 2pt}
\titlespacing\subsection{0pt}{12pt plus 4pt minus 2pt}{0pt plus 2pt minus 2pt}
\titlespacing\subsubsection{0pt}{12pt plus 4pt minus 2pt}{0pt plus 2pt minus 2pt}


\newenvironment{hint}[1][]
  {\begin{em}\textbf{Hint: }}{\end{em}}

\ifshowsolutions
  \newenvironment{solution}[1][]
    {\par\medskip \begin{mdframed}\textbf{Solution~\Alph{question}#1:} \begin{em}}
    {\end{em}\medskip\end{mdframed}\medskip}
  \newenvironment{subsolution}[1][]
    {\par\medskip \begin{mdframed}\textbf{Solution~\Alph{question}#1.\roman{subquestion}:} \begin{em}}
    {\end{em}\medskip\end{mdframed}\medskip}
\else
  \excludecomment{solution}
  \excludecomment{subsolution}
\fi



%%%%%%%%%%%%%%%%%%%%%%%%%%%%%%
% HEADER
%%%%%%%%%%%%%%%%%%%%%%%%%%%%%%

\chead{
  {\vbox{
      \vspace{2mm}
      \large
      Machine Learning \& Data Mining \hfill
      Caltech CS/CNS/EE 155 \hfill \\[1pt]
      Set 3\hfill
      January $20^\text{th}$, 2021 \\
    }
  }
}

\begin{document}
\pagestyle{fancy}



%%%%%%%%%%%%%%%%%%%%%%%%%%%%%%
% POLICIES
%%%%%%%%%%%%%%%%%%%%%%%%%%%%%%

\section*{Policies}
\begin{itemize}
  \item Due 9 PM, January $27^\text{th}$, via Gradescope.
  \item You are free to collaborate on all of the problems, subject to the collaboration policy stated in the syllabus.
  \item In this course, we will be using Google Colab for code submissions. You will need a Google account.
  \item We ask that you use Python 3 (set that as your Colab runtime's Python version) and sklearn version 0.22 (should be the default version for Python 3 in Colab) for your code, and that you comment your code such that the TAs can follow along and run it without any issues.
\end{itemize}

\section*{Submission Instructions}
\begin{itemize}
   \item You are highly encouraged to use the submission template: \url{https://github.com/lakigigar/Caltech-CS155-2021/blob/main/psets/set3/set3template.tex}
   \item Submit your report as a single .pdf file to Gradescope (entry code N8XV6Z), under "Set 3 Report". 
   \item In the report, \textbf{include any images generated by your code} along with your answers to the questions.
   \item Submit your code by \textbf{sharing a link in your report} to your Google Colab notebook for each problem (see naming instructions below). Make sure to set sharing permissions to at least "Anyone with the link can view". \textbf{Links that can not be run by TAs will not be counted as turned in.} Check your links in an incognito window before submitting to be sure. 
   \item For instructions specifically pertaining to the Gradescope submission process, see \url{https://www.gradescope.com/get_started#student-submission}.

\end{itemize}

\section*{Google Colab Instructions}

For each notebook in the course gitub \url{https://github.com/lakigigar/Caltech-CS155-2021}, you need to save a copy to your drive.

\begin{enumerate}
	\item Open the github preview of the notebook, and click the icon to open the colab preview.
	\item On the colab preview, go to File $\rightarrow$ Save a copy in Drive.
	\item Edit your file name to “\url{lastname_firstname_originaltitle}”, e.g.”\url{devasenapathy_kriti_set3-prob2.ipynb}”
\end{enumerate}

%%%%%%%%%%%%%%%%%%%%%%%%%%%%%%
% PROBLEM 1
%%%%%%%%%%%%%%%%%%%%%%%%%%%%%%

\newpage
\section{Decision Trees [30 Points]}


\problem[7]
\begin{solution}
\end{solution}

\problem[4]
\begin{solution}
\end{solution}

\problem[15]
\begin{solution}
\end{solution}

\problem[4]
\begin{solution}
\end{solution}


\newpage


\section{Overfitting Decision Trees [20 Points, 10 EC Points]}

\indent\problem[7] % indent for consistency
\begin{solution}
\end{solution}

\problem[3]
\begin{solution}
\end{solution}

\problem[2]
\begin{solution}
\end{solution}

\problem[3]
\begin{solution}
\end{solution}

\problem[5]
\begin{solution}
\end{solution}

\problem[7 EC]
\begin{solution}
\end{solution}

\problem[2 EC]
\begin{solution}
\end{solution}

\problem[1 EC]
\begin{solution}
\end{solution}



\newpage
\section{The AdaBoost Algorithm [20 Points 20 EC Points]}

\problem[3 EC]
\begin{solution}
\end{solution}

\problem[3 EC]
\begin{solution}
\end{solution}

\problem[2 EC]
\begin{solution}
\end{solution}

\problem[5 EC]
\begin{solution}
\end{solution}

\problem[5 EC]
\begin{solution}
\end{solution}

\problem[2 EC]
\begin{solution}
\end{solution}

\problem[14]
\begin{solution}
\end{solution}

\problem[2]
\begin{solution}
\end{solution}

\problem[2]
\begin{solution}
\end{solution}

\problem[2]
\begin{solution}
\end{solution}

\end{document}